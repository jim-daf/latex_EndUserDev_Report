%%
%% This is file `sample-manuscript.tex',
%% generated with the docstrip utility.
%%
%% The original source files were:
%%
%% samples.dtx  (with options: `all,proceedings,bibtex,manuscript')
%% 
%% IMPORTANT NOTICE:
%% 
%% For the copyright see the source file.
%% 
%% Any modified versions of this file must be renamed
%% with new filenames distinct from sample-manuscript.tex.
%% 
%% For distribution of the original source see the terms
%% for copying and modification in the file samples.dtx.
%% 
%% This generated file may be distributed as long as the
%% original source files, as listed above, are part of the
%% same distribution. (The sources need not necessarily be
%% in the same archive or directory.)
%%
%%
%% Commands for TeXCount
%TC:macro \cite [option:text,text]
%TC:macro \citep [option:text,text]
%TC:macro \citet [option:text,text]
%TC:envir table 0 1
%TC:envir table* 0 1
%TC:envir tabular [ignore] word
%TC:envir displaymath 0 word
%TC:envir math 0 word
%TC:envir comment 0 0
%%
%% The first command in your LaTeX source must be the \documentclass
%% command.
%%
%% For submission and review of your manuscript please change the
%% command to \documentclass[manuscript, screen, review]{acmart}.
%%
%% When submitting camera ready or to TAPS, please change the command
%% to \documentclass[sigconf]{acmart} or whichever template is required
%% for your publication.
%%
%%
\documentclass[manuscript,screen,review]{acmart}
%%
%% \BibTeX command to typeset BibTeX logo in the docs
\AtBeginDocument{%
  \providecommand\BibTeX{{%
    Bib\TeX}}}

%% Rights management information.  This information is sent to you
%% when you complete the rights form.  These commands have SAMPLE
%% values in them; it is your responsibility as an author to replace
%% the commands and values with those provided to you when you
%% complete the rights form.
\setcopyright{acmlicensed}
\copyrightyear{2018}
\acmYear{2018}
\acmDOI{XXXXXXX.XXXXXXX}
%% These commands are for a PROCEEDINGS abstract or paper.
\acmConference[Conference acronym 'XX]{Make sure to enter the correct
  conference title from your rights confirmation email}{June 03--05,
  2018}{Woodstock, NY}
%%
%%  Uncomment \acmBooktitle if the title of the proceedings is different
%%  from ``Proceedings of ...''!
%%
%%\acmBooktitle{Woodstock '18: ACM Symposium on Neural Gaze Detection,
%%  June 03--05, 2018, Woodstock, NY}
\acmISBN{978-1-4503-XXXX-X/2018/06}


%%
%% Submission ID.
%% Use this when submitting an article to a sponsored event. You'll
%% receive a unique submission ID from the organizers
%% of the event, and this ID should be used as the parameter to this command.
%%\acmSubmissionID{123-A56-BU3}

%%
%% For managing citations, it is recommended to use bibliography
%% files in BibTeX format.
%%
%% You can then either use BibTeX with the ACM-Reference-Format style,
%% or BibLaTeX with the acmnumeric or acmauthoryear sytles, that include
%% support for advanced citation of software artefact from the
%% biblatex-software package, also separately available on CTAN.
%%
%% Look at the sample-*-biblatex.tex files for templates showcasing
%% the biblatex styles.
%%

%%
%% The majority of ACM publications use numbered citations and
%% references.  The command \citestyle{authoryear} switches to the
%% "author year" style.
%%
%% If you are preparing content for an event
%% sponsored by ACM SIGGRAPH, you must use the "author year" style of
%% citations and references.
%% Uncommenting
%% the next command will enable that style.
%%\citestyle{acmauthoryear}


%%
%% end of the preamble, start of the body of the document source.
\begin{document}

%%
%% The "title" command has an optional parameter,
%% allowing the author to define a "short title" to be used in page headers.
\title{Title 3}

%%
%% The "author" command and its associated commands are used to define
%% the authors and their affiliations.
%% Of note is the shared affiliation of the first two authors, and the
%% "authornote" and "authornotemark" commands
%% used to denote shared contribution to the research.
\author{Anne-Marie Rommerdahl}
\affiliation{%
 \institution{SDU}
 \city{Odense}
 \country{Denmark}}
\email{anrom25@student.sdu.dk}


\author{Jeremy Alexander Ramírez Galeotti}
\affiliation{%
 \institution{SDU}
 \city{Odense}
 \country{Denmark}}
\email{jeram25@student.sdu.dk}

\author{Dimitrios Dafnis}
\affiliation{%
 \institution{SDU}
 \city{Odense}
 \country{Denmark}}
\email{didaf25@student.sdu.dk}

\author{Nasifa Akter}
\affiliation{%
 \institution{SDU}
 \city{Copenhagen}
 \country{Denmark}}
\email{naakt23@student.sdu.dk}

\author{Mohammad Hosein Kardouni}
\affiliation{%
 \institution{SDU}
 \city{Odense}
 \country{Denmark}}
\email{mokar25@student.sdu.dk}

\author{Ben Trovato}
\authornote{Both authors contributed equally to this research.}
\email{trovato@corporation.com}
\orcid{1234-5678-9012}
\author{G.K.M. Tobin}
\authornotemark[1]
\email{webmaster@marysville-ohio.com}
\affiliation{%
  \institution{Institute for Clarity in Documentation}
  \city{Dublin}
  \state{Ohio}
  \country{USA}
}

\author{Lars Th{\o}rv{\"a}ld}
\affiliation{%
  \institution{The Th{\o}rv{\"a}ld Group}
  \city{Hekla}
  \country{Iceland}}
\email{larst@affiliation.org}

\author{Valerie B\'eranger}
\affiliation{%
  \institution{Inria Paris-Rocquencourt}
  \city{Rocquencourt}
  \country{France}
}

%%
%% By default, the full list of authors will be used in the page
%% headers. Often, this list is too long, and will overlap
%% other information printed in the page headers. This command allows
%% the author to define a more concise list
%% of authors' names for this purpose.
\renewcommand{\shortauthors}{Trovato et al.}

%%
%% The abstract is a short summary of the work to be presented in the
%% article.
\begin{abstract}
  A clear and well-documented \LaTeX\ document is presented as an
  article formatted for publication by ACM in a conference proceedings
  or journal publication. Based on the ``acmart'' document class, this
  article presents and explains many of the common variations, as well
  as many of the formatting elements an author may use in the
  preparation of the documentation of their work.
\end{abstract}

%%
%% The code below is generated by the tool at http://dl.acm.org/ccs.cfm.
%% Please copy and paste the code instead of the example below.
%%
\begin{CCSXML}
<ccs2012>
 <concept>
  <concept_id>00000000.0000000.0000000</concept_id>
  <concept_desc>Do Not Use This Code, Generate the Correct Terms for Your Paper</concept_desc>
  <concept_significance>500</concept_significance>
 </concept>
 <concept>
  <concept_id>00000000.00000000.00000000</concept_id>
  <concept_desc>Do Not Use This Code, Generate the Correct Terms for Your Paper</concept_desc>
  <concept_significance>300</concept_significance>
 </concept>
 <concept>
  <concept_id>00000000.00000000.00000000</concept_id>
  <concept_desc>Do Not Use This Code, Generate the Correct Terms for Your Paper</concept_desc>
  <concept_significance>100</concept_significance>
 </concept>
 <concept>
  <concept_id>00000000.00000000.00000000</concept_id>
  <concept_desc>Do Not Use This Code, Generate the Correct Terms for Your Paper</concept_desc>
  <concept_significance>100</concept_significance>
 </concept>
</ccs2012>
\end{CCSXML}

\ccsdesc[500]{Do Not Use This Code~Generate the Correct Terms for Your Paper}
\ccsdesc[300]{Do Not Use This Code~Generate the Correct Terms for Your Paper}
\ccsdesc{Do Not Use This Code~Generate the Correct Terms for Your Paper}
\ccsdesc[100]{Do Not Use This Code~Generate the Correct Terms for Your Paper}

%%
%% Keywords. The author(s) should pick words that accurately describe
%% the work being presented. Separate the keywords with commas.
\keywords{Do, Not, Use, This, Code, Put, the, Correct, Terms, for,
  Your, Paper}

\received{20 February 2007}
\received[revised]{12 March 2009}
\received[accepted]{5 June 2009}

%%
%% This command processes the author and affiliation and title
%% information and builds the first part of the formatted document.
\maketitle

\section{Introduction}
ACM's consolidated article template, introduced in 2017, provides a
consistent \LaTeX\ style for use across ACM publications, and
incorporates accessibility and metadata-extraction functionality
necessary for future Digital Library endeavors. Numerous ACM and
SIG-specific \LaTeX\ templates have been examined, and their unique
features incorporated into this single new template.

If you are new to publishing with ACM, this document is a valuable
guide to the process of preparing your work for publication. If you
have published with ACM before, this document provides insight and
instruction into more recent changes to the article template.

The ``\verb|acmart|'' document class can be used to prepare articles
for any ACM publication --- conference or journal, and for any stage
of publication, from review to final ``camera-ready'' copy, to the
author's own version, with {\itshape very} few changes to the source.

\section{Background and Related Work}
%% ---- Background ----
%% 1 Background abt. reuse. What is it, and why is it important? (anne)
%% 2 What is the current state of 'reuse' for low-code? Anne paragraph (anne)
%% 3 In what ways have different low-code tools tackled 'reuse'? We look at different low-code tools, not just block-based ones. What are the strengths and weaknesses of their approach to reuse? MAKE SURE TO MENTION TOOLS BY NAME! (dimitris)
%% 4 Now we look at block-based tools for robots (for example, scratch, VEXcode GO, OpenRoberta). How do they tackle reuse? (jeremy + Mohamad)
%% 5 Based on the identified weaknesses in current practices for reuse, we propose our own idea (explain how it's gonna work) (Nasifa)

%% ----Re-done intro----
Software reuse is a broad term, that refers to the practice of reusing previosuly written code, rather than coding from scratch. It is one of the key practices of software engineering. It is in fact such an important part of software engineering, that one of the ways to measure the quality of software is by it's 'Reusability'\cite{SoftwareArchitectureInPractice} - i.e. the degree to which the application or its components can be reused. 
There are many different ways to do reuse in software engineering. Software libraries and frameworks are good examples of software that are intended to be reused. Developers may also scour the internet for things such as open-source software, or code snippets from websites like StackOverFlow, which can be reused. 

There are multiple benefits to software reuse, depending on how the reuse is performed. One example is saving time. Not only can the developer avoid spending time writing the syntax of the code, they may also be able to avoid figuring out the logic of the software, and testing the reused software (assuming the software is tested by its creator). Another benefit is found through modularity. By breaking down a software system into smaller modules, the logic beind features or functions can be contained within a module, and can be tested thoroughly.

Despite reuse being an important practice in software engineering, there is still a limited focus on this practice when it comes to low-code development platforms (LCDP). This lack of reuse focus can easily impact the so-called 'Citizen Developers', who have little or no coding knowledge, and may thus miss out on the benefits of reuse. 
%%--from here: what reuse is there for low-code? What research? IN GENERAL
%%-- Then, what tools? examples of how tools tackle reuse
%%-- Then, what tools for blocks and robots? How do they tackle reuse? What are their strengths/weaknesses/limitations? MAKE SURE TO MENTION BY NAME!!
%%-- Show the tool we'll be working on (what does it do for reuse, and weaknesses).
%%-- Based on all that, what is our idea? What gap is it, we're addressing and how?
%% ---------------------
A study from 2021 studied several low-code platforms (LCPs), in order to identify characteristic features of LCPs. 
The identified features were presented according to how frequent they occured, with domain-specific reference artifacts being 
categorized as 'rare'. Most studied systems offered catalogs of "reusable functions or examples of predefined processes", 
but they were found to be generic, or have a limited scope\cite{LowCodePlatform}. 
%%Maybe move SevenDevPlatforms up here?
There have been proposed some ideas on how to promote reuse for LCPs, such as the strongly-typed rich templating language OSTRICH, developed for the model-driven low-code platform OutSystems. OutSystems provides scaffolding mechanisms for common development patterns and sample screen templates, both designed by experts on domain-specific languages (DSL).
The practice of using templates in the OutSystems platform involves cloning and modifying samples, which may require more knowledge than the end-user posseses.
The goal of OSTRICH is to remove this need for adapation when using templates, to remove the knowledge-barrier when making use of the available templates. This is done by abstracting and parameterizing the templates. 
A limitation of OSTRICH, is that it currently only supports the top nine most used production-ready screen templates from OutSystems. The end-user may not create and save their own templates, nor can they re-apply a template which they have customized. 
% type of experts NOT defined! page 1

Another approach focused on enabling model reuse by converting and merging heterogeneous models together into several graphs, which are then merged into one single graph (The Knowledge Graph), which acts as the repository of models. The Knowledge Graph can be queried to predict the next modeling step, based on the model being constructed by the user. This approach focuses on how to store, query, recommend and integrate the pre-defined models effeciently. End-Users can also persist their own models to the repository for later reuse.\\
For citizen developers, this feature of recommending models which have been constructed by domain experts and then developed by model experts could prove very useful. However, while the user may persist their own models, the study is clearly not focused on guiding the user towards reusing their own models.\\ 
On the other hand, some existing LCDPs offer the user the ability to create their own models - for example by defining a new block in a block-based tool\cite{ScratchExample}.

%having a graph serve as the model repository, created by converting and merging heterogeneous models together\cite{ReuseKnowledgeGraphs}. This approach aims to enable model reuse by merging the graphs into a single graph (the Knowledge graph), and then querying the Knowledge Graph to predict the next modeling step, based on the model being constructed by the user. This approach focuses on how to store, query, recommend and integrate the pre-defined models effeciently. End-Users can also persist their own models to the repository for later reuse. 
%Low code platform tools that focus on reuse
Building on the ideas discussed for improving reuse in low-code development platforms (LCDPs), 
several popular tools show these concepts in action. 
For instance, Webflow\cite{webflow} is a leading low-code platform that offers a wealth of features for building responsive websites. 
One of its standout features is the ability to create reusable components and UI kits, which can significantly 
speed up the development process. With Webflow’s intuitive interface, 
developers can quickly design and prototype components, and then reuse them across multiple pages and projects. 

In a similar way, Mendix\cite{mendix} takes this further for full enterprise apps by offering 
shareable building blocks like simple actions (microflows) and UI parts that anyone 
on a team can grab and use again without recoding. Through its Marketplace, a free online hub, 
you can download ready templates, connectors for tools like Salesforce, and basic setups 
that fit right into new projects, making everything faster and more uniform. 
This approach builds on the flexibility seen in platforms like Webflow, but adds 
strong team tools and AI suggestions to spot and create reusable pieces, empowering 
even beginners to build complex apps while keeping reuse simple and widespread.

OutSystems\cite{outsystems} further enhances the concept of reuse in low-code development platforms by emphasizing rapid application delivery through its robust set of features.
Like Webflow and Mendix, OutSystems also provides a library of reusable components and templates that help developers complete projects faster. 
Its user-friendly visual development environment allows users to easily drag and drop elements while connecting with existing systems. 
OutSystems also supports teamwork with built-in version control and feedback features, making it easy for teams to share and improve 
reusable components. Additionally, the platform uses AI to suggest the best solutions and components for specific tasks, helping to 
streamline the development process. By encouraging reuse at both individual and team levels, OutSystems enables organizations to create 
scalable applications quickly while ensuring quality and consistency.

In order to analyze how block-based robotics environments address reuse area, 4 representative platforms were compared: mBlock, MakeCode, SPIKE LEGO, VEXcode GO and Open Roberta. The comparison focused on three main dimensions of reuse: structural reuse (through user-defined blocks or functions), social reuse (through sharing or remixing existing projects), and interoperable reuse (through import/export capabilities).

\begin{table*}
  \small   % reduce el tamaño de la fuente
  \caption{Block Based Robotics Environments Reuse Support}
  \label{tab:reuse_support}
  \begin{tabular}{lcccc}
    \toprule
    Platform & Structural Reuse & Social Reuse & Interoperable Reuse & Reuse Support \\
    \midrule
    VEXcode GO    & X & X &  & Medium \\
    mBlock        & X & X & X & Medium \\
    MakeCode      & X & X & X & Medium \\
    Spike Lego    & X &  & X & Low \\
    Open Roberta  &  & X &  & Low \\
    \bottomrule
  \end{tabular}
\end{table*}


In this context, “reuse support” represents a scale that measures how effectively each platform facilitates reuse-related features. High reuse support indicates that users can easily create, share, and adapt existing components or projects. Medium reuse support suggests that some reuse mechanisms are available but limited in scope or flexibility. Low reuse support implies that the platform provides only minimal or restricted features to promote reuse and improve user productivity.

As shown in Table 1, although these platforms include reusability features, they are quite limited, as none of them provide users with clear guidance on how to use these tools effectively, which restricts their ability to fully leverage them.



Despite all of the useful features that these tools have, none of them 
provides guidance to the end-users to create custom reusable components which is the key feature of our project.




Research also indicates that block based programming 
environments should guide the end users towards good code organization as many may lack the necessary knowledge or may become 
stuck due to errors.\cite{ChallengesInBlockBasedEnvironments} Although block based programming tools like Blockly
were invented to teach programming to beginners by simple
examples, Mayr-Dorn et al. mention that it is possible to express even
large and highly complex real-world robot programs with the
language concepts offered by these kind of block-based tools. \cite{blockBasedLanguagesForRobotProgramming} 



\section{Study Design}


\subsection{Problem Investigation}
\subsection{Treatment Design}

Our treatment focuses on developing a guided reuse assistant for the OpenRoberta Lab environment.
The purpose of this tool is to help users recognize when parts of their robot programs can be reused, 
and to make it easier for them to create reusable custom blocks. By doing this, we aim to reduce 
repetitive code and help users learn important programming concepts such as modularity and abstraction.

\subsubsection{Overview of the Tool}

The guided reuse assistant is built as an extension inside Open Roberta Lab, which uses the Blockly 
framework. The assistant runs directly in the web browser and interacts with the user’s block workspace. 
Its main job is to look through the user’s program, find repeated sequences of blocks, and guide 
the user in turning them into reusable blocks.

The tool works in three main steps:
\begin{enumerate}
    \item \textbf{Detecting Repeated Code:} 
    The assistant automatically scans the user’s program and searches for parts that look the same
     or very similar. These are marked as potential duplicates.

    \item \textbf{Highlighting and Suggesting Reuse:} 
    Once duplicates are found, the system highlights them in the workspace and shows a message 
    suggesting that these sections could be made into a reusable block (function). 
    This helps users see repetition they might not have noticed before.

    \item \textbf{Helping the User Create a New Block:} 
    If the user agrees to the suggestion, the assistant opens a small guide to help them create 
    the new block. It automatically detects any small differences between the repeated parts, 
    such as numbers or variable names, and turns them into inputs (parameters) for the new block. 
    When the block is created, all the repeated code is replaced by calls to this new reusable block.
\end{enumerate}

\subsection{Treatment Validation}
The treatment validation for this study adopts a mixed-methods evaluation approach to assess 
the effectiveness of the proposed features for guiding users in creating reusable custom 
blocks within the OpenRoberta environment. Participants will be recruited from local 
educational institutions, specifically chemistry students and teachers who frequently engage 
in laboratory work. A sufficient number of participants will be selected to ensure a diverse 
range of experience levels with block-based programming.
The experimental setup will take place in a controlled environment, where participants will 
be divided into two groups: one using the enhanced OpenRoberta platform with guided block 
creation features, and the other using the standard version without these enhancements. 
The procedure will begin with a pre-test to evaluate participants’ prior understanding of 
modular programming concepts, followed by a series of tasks in which they will create 
reusable blocks from given code segments. Participants’ interactions with the platform will 
be observed throughout the experiment.
Data collection will include both quantitative measures, such as task completion time and 
accuracy in creating reusable blocks and qualitative feedback obtained through post-task 
interview. The analysis will compare performance metrics between the two groups 
and apply thematic analysis to the qualitative data to identify user experiences and 
perceptions of the new features’ usability and effectiveness.
This comprehensive evaluation will provide a detailed understanding of how useful and effective 
is the block creation guidance feature to the end-users.

Existing block-based environments provide mechanisms for reuse, but lack intelligent support to help users recognize and apply reuse in practice.

To address this gap, our project introduces a guided reuse assistant within the Open Roberta Lab environment.
The tool is designed to help users identify and apply reuse more easily while creating their robot programs. It works by automatically scanning a user’s block-based program to detect repeated code segments that appear in different parts of the workspace. Once these duplicates are found, the system highlights them visually, drawing the user’s attention to patterns that could be simplified.

When repeated blocks are detected, the assistant suggests creating a reusable custom block (function). It then helps the user generate this new block by identifying the small differences between the repeated parts—such as numbers, variables, or parameters—and turning these differences into inputs for the new block. After the user confirms, the system automatically replaces all the repeated sequences with calls to the newly created reusable block.

By combining ideas from procedural abstraction (organizing code into meaningful, reusable parts) and automated refactoring (improving code through intelligent transformations), our tool aims to make block-based programming more structured and efficient.
It encourages users to build programs that are modular and easier to maintain, helps reduce unnecessary repetition, and supports learning by making the concept of reuse clear and hands-on.

In summary, our work bridges the gap between existing theoretical approaches to software reuse and their real-world application in block-based programming environments. Through this guided and semi-automated approach, we aim to make reuse visible, understandable, and practical for end-users working in Open Roberta.


\section{Modifications}

Modifying the template --- including but not limited to: adjusting
margins, typeface sizes, line spacing, paragraph and list definitions,
and the use of the \verb|\vspace| command to manually adjust the
vertical spacing between elements of your work --- is not allowed.

{\bfseries Your document will be returned to you for revision if
  modifications are discovered.}

\section{Typefaces}

The ``\verb|acmart|'' document class requires the use of the
``Libertine'' typeface family. Your \TeX\ installation should include
this set of packages. Please do not substitute other typefaces. The
``\verb|lmodern|'' and ``\verb|ltimes|'' packages should not be used,
as they will override the built-in typeface families.

\section{Title Information}

The title of your work should use capital letters appropriately -
\url{https://capitalizemytitle.com/} has useful rules for
capitalization. Use the {\verb|title|} command to define the title of
your work. If your work has a subtitle, define it with the
{\verb|subtitle|} command.  Do not insert line breaks in your title.

If your title is lengthy, you must define a short version to be used
in the page headers, to prevent overlapping text. The \verb|title|
command has a ``short title'' parameter:
\begin{verbatim}
  \title[short title]{full title}
\end{verbatim}

\section{Authors and Affiliations}

Each author must be defined separately for accurate metadata
identification.  As an exception, multiple authors may share one
affiliation. Authors' names should not be abbreviated; use full first
names wherever possible. Include authors' e-mail addresses whenever
possible.

Grouping authors' names or e-mail addresses, or providing an ``e-mail
alias,'' as shown below, is not acceptable:
\begin{verbatim}
  \author{Brooke Aster, David Mehldau}
  \email{dave,judy,steve@university.edu}
  \email{firstname.lastname@phillips.org}
\end{verbatim}

The \verb|authornote| and \verb|authornotemark| commands allow a note
to apply to multiple authors --- for example, if the first two authors
of an article contributed equally to the work.

If your author list is lengthy, you must define a shortened version of
the list of authors to be used in the page headers, to prevent
overlapping text. The following command should be placed just after
the last \verb|\author{}| definition:
\begin{verbatim}
  \renewcommand{\shortauthors}{McCartney, et al.}
\end{verbatim}
Omitting this command will force the use of a concatenated list of all
of the authors' names, which may result in overlapping text in the
page headers.

The article template's documentation, available at
\url{https://www.acm.org/publications/proceedings-template}, has a
complete explanation of these commands and tips for their effective
use.

Note that authors' addresses are mandatory for journal articles.

\section{Rights Information}

Authors of any work published by ACM will need to complete a rights
form. Depending on the kind of work, and the rights management choice
made by the author, this may be copyright transfer, permission,
license, or an OA (open access) agreement.

Regardless of the rights management choice, the author will receive a
copy of the completed rights form once it has been submitted. This
form contains \LaTeX\ commands that must be copied into the source
document. When the document source is compiled, these commands and
their parameters add formatted text to several areas of the final
document:
\begin{itemize}
\item the ``ACM Reference Format'' text on the first page.
\item the ``rights management'' text on the first page.
\item the conference information in the page header(s).
\end{itemize}

Rights information is unique to the work; if you are preparing several
works for an event, make sure to use the correct set of commands with
each of the works.

The ACM Reference Format text is required for all articles over one
page in length, and is optional for one-page articles (abstracts).

\section{CCS Concepts and User-Defined Keywords}

Two elements of the ``acmart'' document class provide powerful
taxonomic tools for you to help readers find your work in an online
search.

The ACM Computing Classification System ---
\url{https://www.acm.org/publications/class-2012} --- is a set of
classifiers and concepts that describe the computing
discipline. Authors can select entries from this classification
system, via \url{https://dl.acm.org/ccs/ccs.cfm}, and generate the
commands to be included in the \LaTeX\ source.

User-defined keywords are a comma-separated list of words and phrases
of the authors' choosing, providing a more flexible way of describing
the research being presented.

CCS concepts and user-defined keywords are required for for all
articles over two pages in length, and are optional for one- and
two-page articles (or abstracts).

\section{Sectioning Commands}

Your work should use standard \LaTeX\ sectioning commands:
\verb|\section|, \verb|\subsection|, \verb|\subsubsection|,
\verb|\paragraph|, and \verb|\subparagraph|. The sectioning levels up to
\verb|\subsusection| should be numbered; do not remove the numbering
from the commands.

Simulating a sectioning command by setting the first word or words of
a paragraph in boldface or italicized text is {\bfseries not allowed.}

Below are examples of sectioning commands.

\subsection{Subsection}
\label{sec:subsection}

This is a subsection.

\subsubsection{Subsubsection}
\label{sec:subsubsection}

This is a subsubsection.

\paragraph{Paragraph}

This is a paragraph.

\subparagraph{Subparagraph}

This is a subparagraph.

\section{Tables}

The ``\verb|acmart|'' document class includes the ``\verb|booktabs|''
package --- \url{https://ctan.org/pkg/booktabs} --- for preparing
high-quality tables.

Table captions are placed {\itshape above} the table.

Because tables cannot be split across pages, the best placement for
them is typically the top of the page nearest their initial cite.  To
ensure this proper ``floating'' placement of tables, use the
environment \textbf{table} to enclose the table's contents and the
table caption.  The contents of the table itself must go in the
\textbf{tabular} environment, to be aligned properly in rows and
columns, with the desired horizontal and vertical rules.  Again,
detailed instructions on \textbf{tabular} material are found in the
\textit{\LaTeX\ User's Guide}.

Immediately following this sentence is the point at which
Table~\ref{tab:freq} is included in the input file; compare the
placement of the table here with the table in the printed output of
this document.

\begin{table}
  \caption{Frequency of Special Characters}
  \label{tab:freq}
  \begin{tabular}{ccl}
    \toprule
    Non-English or Math&Frequency&Comments\\
    \midrule
    \O & 1 in 1,000& For Swedish names\\
    $\pi$ & 1 in 5& Common in math\\
    \$ & 4 in 5 & Used in business\\
    $\Psi^2_1$ & 1 in 40,000& Unexplained usage\\
  \bottomrule
\end{tabular}
\end{table}

To set a wider table, which takes up the whole width of the page's
live area, use the environment \textbf{table*} to enclose the table's
contents and the table caption.  As with a single-column table, this
wide table will ``float'' to a location deemed more
desirable. Immediately following this sentence is the point at which
Table~\ref{tab:commands} is included in the input file; again, it is
instructive to compare the placement of the table here with the table
in the printed output of this document.

\begin{table*}
  \caption{Some Typical Commands}
  \label{tab:commands}
  \begin{tabular}{ccl}
    \toprule
    Command &A Number & Comments\\
    \midrule
    \texttt{{\char'134}author} & 100& Author \\
    \texttt{{\char'134}table}& 300 & For tables\\
    \texttt{{\char'134}table*}& 400& For wider tables\\
    \bottomrule
  \end{tabular}
\end{table*}

Always use midrule to separate table header rows from data rows, and
use it only for this purpose. This enables assistive technologies to
recognise table headers and support their users in navigating tables
more easily.

\section{Math Equations}
You may want to display math equations in three distinct styles:
inline, numbered or non-numbered display.  Each of the three are
discussed in the next sections.

\subsection{Inline (In-text) Equations}
A formula that appears in the running text is called an inline or
in-text formula.  It is produced by the \textbf{math} environment,
which can be invoked with the usual
\texttt{{\char'134}begin\,\ldots{\char'134}end} construction or with
the short form \texttt{\$\,\ldots\$}. You can use any of the symbols
and structures, from $\alpha$ to $\omega$, available in
\LaTeX~\cite{Lamport:LaTeX}; this section will simply show a few
examples of in-text equations in context. Notice how this equation:
\begin{math}
  \lim_{n\rightarrow \infty}x=0
\end{math},
set here in in-line math style, looks slightly different when
set in display style.  (See next section).

\subsection{Display Equations}
A numbered display equation---one set off by vertical space from the
text and centered horizontally---is produced by the \textbf{equation}
environment. An unnumbered display equation is produced by the
\textbf{displaymath} environment.

Again, in either environment, you can use any of the symbols and
structures available in \LaTeX\@; this section will just give a couple
of examples of display equations in context.  First, consider the
equation, shown as an inline equation above:
\begin{equation}
  \lim_{n\rightarrow \infty}x=0
\end{equation}
Notice how it is formatted somewhat differently in
the \textbf{displaymath}
environment.  Now, we'll enter an unnumbered equation:
\begin{displaymath}
  \sum_{i=0}^{\infty} x + 1
\end{displaymath}
and follow it with another numbered equation:
\begin{equation}
  \sum_{i=0}^{\infty}x_i=\int_{0}^{\pi+2} f
\end{equation}
just to demonstrate \LaTeX's able handling of numbering.

\section{Figures}

The ``\verb|figure|'' environment should be used for figures. One or
more images can be placed within a figure. If your figure contains
third-party material, you must clearly identify it as such, as shown
in the example below.
\begin{figure}[h]
  \centering
  \includegraphics[width=\linewidth]{sample-franklin}
  \caption{1907 Franklin Model D roadster. Photograph by Harris \&
    Ewing, Inc. [Public domain], via Wikimedia
    Commons. (\url{https://goo.gl/VLCRBB}).}
  \Description{A woman and a girl in white dresses sit in an open car.}
\end{figure}

Your figures should contain a caption which describes the figure to
the reader.

Figure captions are placed {\itshape below} the figure.

Every figure should also have a figure description unless it is purely
decorative. These descriptions convey what’s in the image to someone
who cannot see it. They are also used by search engine crawlers for
indexing images, and when images cannot be loaded.

A figure description must be unformatted plain text less than 2000
characters long (including spaces).  {\bfseries Figure descriptions
  should not repeat the figure caption – their purpose is to capture
  important information that is not already provided in the caption or
  the main text of the paper.} For figures that convey important and
complex new information, a short text description may not be
adequate. More complex alternative descriptions can be placed in an
appendix and referenced in a short figure description. For example,
provide a data table capturing the information in a bar chart, or a
structured list representing a graph.  For additional information
regarding how best to write figure descriptions and why doing this is
so important, please see
\url{https://www.acm.org/publications/taps/describing-figures/}.

\subsection{The ``Teaser Figure''}

A ``teaser figure'' is an image, or set of images in one figure, that
are placed after all author and affiliation information, and before
the body of the article, spanning the page. If you wish to have such a
figure in your article, place the command immediately before the
\verb|\maketitle| command:
\begin{verbatim}
  \begin{teaserfigure}
    \includegraphics[width=\textwidth]{sampleteaser}
    \caption{figure caption}
    \Description{figure description}
  \end{teaserfigure}
\end{verbatim}

\section{Citations and Bibliographies}

The use of \BibTeX\ for the preparation and formatting of one's
references is strongly recommended. Authors' names should be complete
--- use full first names (``Donald E. Knuth'') not initials
(``D. E. Knuth'') --- and the salient identifying features of a
reference should be included: title, year, volume, number, pages,
article DOI, etc.

The bibliography is included in your source document with these two
commands, placed just before the \verb|\end{document}| command:
\begin{verbatim}
  \bibliographystyle{ACM-Reference-Format}
  \bibliography{bibfile}
\end{verbatim}
where ``\verb|bibfile|'' is the name, without the ``\verb|.bib|''
suffix, of the \BibTeX\ file.

Citations and references are numbered by default. A small number of
ACM publications have citations and references formatted in the
``author year'' style; for these exceptions, please include this
command in the {\bfseries preamble} (before the command
``\verb|\begin{document}|'') of your \LaTeX\ source:
\begin{verbatim}
  \citestyle{acmauthoryear}
\end{verbatim}


  Some examples.  A paginated journal article \cite{Abril07}, an
  enumerated journal article \cite{Cohen07}, a reference to an entire
  issue \cite{JCohen96}, a monograph (whole book) \cite{Kosiur01}, a
  monograph/whole book in a series (see 2a in spec. document)
  \cite{Harel79}, a divisible-book such as an anthology or compilation
  \cite{Editor00} followed by the same example, however we only output
  the series if the volume number is given \cite{Editor00a} (so
  Editor00a's series should NOT be present since it has no vol. no.),
  a chapter in a divisible book \cite{Spector90}, a chapter in a
  divisible book in a series \cite{Douglass98}, a multi-volume work as
  book \cite{Knuth97}, a couple of articles in a proceedings (of a
  conference, symposium, workshop for example) (paginated proceedings
  article) \cite{Andler79, Hagerup1993}, a proceedings article with
  all possible elements \cite{Smith10}, an example of an enumerated
  proceedings article \cite{VanGundy07}, an informally published work
  \cite{Harel78}, a couple of preprints \cite{Bornmann2019,
    AnzarootPBM14}, a doctoral dissertation \cite{Clarkson85}, a
  master's thesis: \cite{anisi03}, an online document / world wide web
  resource \cite{Thornburg01, Ablamowicz07, Poker06}, a video game
  (Case 1) \cite{Obama08} and (Case 2) \cite{Novak03} and \cite{Lee05}
  and (Case 3) a patent \cite{JoeScientist001}, work accepted for
  publication \cite{rous08}, 'YYYYb'-test for prolific author
  \cite{SaeediMEJ10} and \cite{SaeediJETC10}. Other cites might
  contain 'duplicate' DOI and URLs (some SIAM articles)
  \cite{Kirschmer:2010:AEI:1958016.1958018}. Boris / Barbara Beeton:
  multi-volume works as books \cite{MR781536} and \cite{MR781537}. A
  presentation~\cite{Reiser2014}. An article under
  review~\cite{Baggett2025}. A
  couple of citations with DOIs:
  \cite{2004:ITE:1009386.1010128,Kirschmer:2010:AEI:1958016.1958018}. Online
  citations: \cite{TUGInstmem, Thornburg01, CTANacmart}.
  Artifacts: \cite{R} and \cite{UMassCitations}.

\section{Acknowledgments}

Identification of funding sources and other support, and thanks to
individuals and groups that assisted in the research and the
preparation of the work should be included in an acknowledgment
section, which is placed just before the reference section in your
document.

This section has a special environment:
\begin{verbatim}
  \begin{acks}
  ...
  \end{acks}
\end{verbatim}
so that the information contained therein can be more easily collected
during the article metadata extraction phase, and to ensure
consistency in the spelling of the section heading.

Authors should not prepare this section as a numbered or unnumbered {\verb|\section|}; please use the ``{\verb|acks|}'' environment.

\section{Appendices}

If your work needs an appendix, add it before the
``\verb|\end{document}|'' command at the conclusion of your source
document.

Start the appendix with the ``\verb|appendix|'' command:
\begin{verbatim}
  \appendix
\end{verbatim}
and note that in the appendix, sections are lettered, not
numbered. This document has two appendices, demonstrating the section
and subsection identification method.

\section{Multi-language papers}

Papers may be written in languages other than English or include
titles, subtitles, keywords and abstracts in different languages (as a
rule, a paper in a language other than English should include an
English title and an English abstract).  Use \verb|language=...| for
every language used in the paper.  The last language indicated is the
main language of the paper.  For example, a French paper with
additional titles and abstracts in English and German may start with
the following command
\begin{verbatim}
\documentclass[sigconf, language=english, language=german,
               language=french]{acmart}
\end{verbatim}

The title, subtitle, keywords and abstract will be typeset in the main
language of the paper.  The commands \verb|\translatedXXX|, \verb|XXX|
begin title, subtitle and keywords, can be used to set these elements
in the other languages.  The environment \verb|translatedabstract| is
used to set the translation of the abstract.  These commands and
environment have a mandatory first argument: the language of the
second argument.  See \verb|sample-sigconf-i13n.tex| file for examples
of their usage.

\section{SIGCHI Extended Abstracts}

The ``\verb|sigchi-a|'' template style (available only in \LaTeX\ and
not in Word) produces a landscape-orientation formatted article, with
a wide left margin. Three environments are available for use with the
``\verb|sigchi-a|'' template style, and produce formatted output in
the margin:
\begin{description}
\item[\texttt{sidebar}:]  Place formatted text in the margin.
\item[\texttt{marginfigure}:] Place a figure in the margin.
\item[\texttt{margintable}:] Place a table in the margin.
\end{description}

%%
%% The acknowledgments section is defined using the "acks" environment
%% (and NOT an unnumbered section). This ensures the proper
%% identification of the section in the article metadata, and the
%% consistent spelling of the heading.
\begin{acks}
To Robert, for the bagels and explaining CMYK and color spaces.
\end{acks}

%%
%% The next two lines define the bibliography style to be used, and
%% the bibliography file.
\bibliographystyle{ACM-Reference-Format}
\bibliography{sample-base}


%%
%% If your work has an appendix, this is the place to put it.
\appendix

\section{Research Methods}

\subsection{Part One}

Lorem ipsum dolor sit amet, consectetur adipiscing elit. Morbi
malesuada, quam in pulvinar varius, metus nunc fermentum urna, id
sollicitudin purus odio sit amet enim. Aliquam ullamcorper eu ipsum
vel mollis. Curabitur quis dictum nisl. Phasellus vel semper risus, et
lacinia dolor. Integer ultricies commodo sem nec semper.

\subsection{Part Two}

Etiam commodo feugiat nisl pulvinar pellentesque. Etiam auctor sodales
ligula, non varius nibh pulvinar semper. Suspendisse nec lectus non
ipsum convallis congue hendrerit vitae sapien. Donec at laoreet
eros. Vivamus non purus placerat, scelerisque diam eu, cursus
ante. Etiam aliquam tortor auctor efficitur mattis.

\section{Online Resources}

Nam id fermentum dui. Suspendisse sagittis tortor a nulla mollis, in
pulvinar ex pretium. Sed interdum orci quis metus euismod, et sagittis
enim maximus. Vestibulum gravida massa ut felis suscipit
congue. Quisque mattis elit a risus ultrices commodo venenatis eget
dui. Etiam sagittis eleifend elementum.

Nam interdum magna at lectus dignissim, ac dignissim lorem
rhoncus. Maecenas eu arcu ac neque placerat aliquam. Nunc pulvinar
massa et mattis lacinia.

\end{document}
\endinput
%%
%% End of file `sample-manuscript.tex'.
