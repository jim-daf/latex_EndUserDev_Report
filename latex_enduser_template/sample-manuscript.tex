%%
%% This is file `sample-manuscript.tex',
%% generated with the docstrip utility.
%%
%% The original source files were:
%%
%% samples.dtx  (with options: `all,proceedings,bibtex,manuscript')
%% 
%% IMPORTANT NOTICE:
%% 
%% For the copyright see the source file.
%% 
%% Any modified versions of this file must be renamed
%% with new filenames distinct from sample-manuscript.tex.
%% 
%% For distribution of the original source see the terms
%% for copying and modification in the file samples.dtx.
%% 
%% This generated file may be distributed as long as the
%% original source files, as listed above, are part of the
%% same distribution. (The sources need not necessarily be
%% in the same archive or directory.)
%%
%%
%% Commands for TeXCount
%TC:macro \cite [option:text,text]
%TC:macro \citep [option:text,text]
%TC:macro \citet [option:text,text]
%TC:envir table 0 1
%TC:envir table* 0 1
%TC:envir tabular [ignore] word
%TC:envir displaymath 0 word
%TC:envir math 0 word
%TC:envir comment 0 0
%%
%% The first command in your LaTeX source must be the \documentclass
%% command.
%%
%% For submission and review of your manuscript please change the
%% command to \documentclass[manuscript, screen, review]{acmart}.
%%
%% When submitting camera ready or to TAPS, please change the command
%% to \documentclass[sigconf]{acmart} or whichever template is required
%% for your publication.
%%
%%
\documentclass[manuscript,screen,review]{acmart}
%%
%% \BibTeX command to typeset BibTeX logo in the docs
\AtBeginDocument{%
  \providecommand\BibTeX{{%
    Bib\TeX}}}

\usepackage{array}

%% Rights management information.  This information is sent to you
%% when you complete the rights form.  These commands have SAMPLE
%% values in them; it is your responsibility as an author to replace
%% the commands and values with those provided to you when you
%% complete the rights form.
\setcopyright{acmlicensed}
\copyrightyear{2018}
\acmYear{2018}
\acmDOI{XXXXXXX.XXXXXXX}
%% These commands are for a PROCEEDINGS abstract or paper.
\acmConference[Conference acronym 'XX]{Make sure to enter the correct
  conference title from your rights confirmation email}{June 03--05,
  2018}{Woodstock, NY}
%%
%%  Uncomment \acmBooktitle if the title of the proceedings is different
%%  from ``Proceedings of ...''!
%%
%%\acmBooktitle{Woodstock '18: ACM Symposium on Neural Gaze Detection,
%%  June 03--05, 2018, Woodstock, NY}
\acmISBN{978-1-4503-XXXX-X/2018/06}


%%
%% Submission ID.
%% Use this when submitting an article to a sponsored event. You'll
%% receive a unique submission ID from the organizers
%% of the event, and this ID should be used as the parameter to this command.
%%\acmSubmissionID{123-A56-BU3}

%%
%% For managing citations, it is recommended to use bibliography
%% files in BibTeX format.
%%
%% You can then either use BibTeX with the ACM-Reference-Format style,
%% or BibLaTeX with the acmnumeric or acmauthoryear sytles, that include
%% support for advanced citation of software artefact from the
%% biblatex-software package, also separately available on CTAN.
%%
%% Look at the sample-*-biblatex.tex files for templates showcasing
%% the biblatex styles.
%%

%%
%% The majority of ACM publications use numbered citations and
%% references.  The command \citestyle{authoryear} switches to the
%% "author year" style.
%%
%% If you are preparing content for an event
%% sponsored by ACM SIGGRAPH, you must use the "author year" style of
%% citations and references.
%% Uncommenting
%% the next command will enable that style.
%%\citestyle{acmauthoryear}


%%
%% end of the preamble, start of the body of the document source.
\begin{document}

%%
%% The "title" command has an optional parameter,
%% allowing the author to define a "short title" to be used in page headers.
\title{Title 3}

%%
%% The "author" command and its associated commands are used to define
%% the authors and their affiliations.
%% Of note is the shared affiliation of the first two authors, and the
%% "authornote" and "authornotemark" commands
%% used to denote shared contribution to the research.
\author{Anne-Marie Rommerdahl}
\affiliation{%
 \institution{SDU}
 \city{Odense}
 \country{Denmark}}
\email{anrom25@student.sdu.dk}

\author{Jeremy Alexander Ramírez Galeotti}
\affiliation{%
 \institution{SDU}
 \city{Odense}
 \country{Denmark}}
\email{jeram25@student.sdu.dk}

\author{Dimitrios Dafnis}
\affiliation{%
 \institution{SDU}
 \city{Odense}
 \country{Denmark}}
\email{didaf25@student.sdu.dk}

\author{Nasifa Akter}
\affiliation{%
 \institution{SDU}
 \city{Copenhagen}
 \country{Denmark}}
\email{naakt23@student.sdu.dk}

\author{Mohammad Hosein Kardouni}
\affiliation{%
 \institution{SDU}
 \city{Odense}
 \country{Denmark}}
\email{mokar25@student.sdu.dk}



%%
%% By default, the full list of authors will be used in the page
%% headers. Often, this list is too long, and will overlap
%% other information printed in the page headers. This command allows
%% the author to define a more concise list
%% of authors' names for this purpose.
\renewcommand{\shortauthors}{Trovato et al.}

%%
%% The abstract is a short summary of the work to be presented in the
%% article.
\begin{abstract}
  A clear and well-documented \LaTeX\ document is presented as an
  article formatted for publication by ACM in a conference proceedings
  or journal publication. Based on the ``acmart'' document class, this
  article presents and explains many of the common variations, as well
  as many of the formatting elements an author may use in the
  preparation of the documentation of their work.
\end{abstract}

%%
%% The code below is generated by the tool at http://dl.acm.org/ccs.cfm.
%% Please copy and paste the code instead of the example below.
%%
\begin{CCSXML}
<ccs2012>
 <concept>
  <concept_id>00000000.0000000.0000000</concept_id>
  <concept_desc>Do Not Use This Code, Generate the Correct Terms for Your Paper</concept_desc>
  <concept_significance>500</concept_significance>
 </concept>
 <concept>
  <concept_id>00000000.00000000.00000000</concept_id>
  <concept_desc>Do Not Use This Code, Generate the Correct Terms for Your Paper</concept_desc>
  <concept_significance>300</concept_significance>
 </concept>
 <concept>
  <concept_id>00000000.00000000.00000000</concept_id>
  <concept_desc>Do Not Use This Code, Generate the Correct Terms for Your Paper</concept_desc>
  <concept_significance>100</concept_significance>
 </concept>
 <concept>
  <concept_id>00000000.00000000.00000000</concept_id>
  <concept_desc>Do Not Use This Code, Generate the Correct Terms for Your Paper</concept_desc>
  <concept_significance>100</concept_significance>
 </concept>
</ccs2012>
\end{CCSXML}

\ccsdesc[500]{Do Not Use This Code~Generate the Correct Terms for Your Paper}
\ccsdesc[300]{Do Not Use This Code~Generate the Correct Terms for Your Paper}
\ccsdesc{Do Not Use This Code~Generate the Correct Terms for Your Paper}
\ccsdesc[100]{Do Not Use This Code~Generate the Correct Terms for Your Paper}

%%
%% Keywords. The author(s) should pick words that accurately describe
%% the work being presented. Separate the keywords with commas.
\keywords{Do, Not, Use, This, Code, Put, the, Correct, Terms, for,
  Your, Paper}

\received{20 February 2007}
\received[revised]{12 March 2009}
\received[accepted]{5 June 2009}

%%
%% This command processes the author and affiliation and title
%% information and builds the first part of the formatted document.
\maketitle

\section{Introduction}


\section{Background and Related Work}
%% ---- Background ----
%% 1 Background abt. reuse. What is it, and why is it important? (anne)
%% 2 What is the current state of 'reuse' for low-code? Anne paragraph (anne)
%% 3 In what ways have different low-code tools tackled 'reuse'? We look at different low-code tools, not just block-based ones. What are the strengths and weaknesses of their approach to reuse? MAKE SURE TO MENTION TOOLS BY NAME! (dimitris)
%% 4 Now we look at block-based tools for robots (for example, scratch, VEXcode GO, OpenRoberta). How do they tackle reuse? (jeremy + Mohamad)
%% 5 Based on the identified weaknesses in current practices for reuse, we propose our own idea (explain how it's gonna work) (Nasifa)

%% ----Re-done intro----
Software reuse is a broad term, that refers to the practice of reusing previously written code, rather than coding from scratch. It is such an important part of software engineering, that one of the ways to measure the quality of software is by it's 'Reusability'\cite{SoftwareArchitectureInPractice} - i.e. the degree to which the application or its components can be reused. 
There are multiple benefits to practicing reuse in software engineering. One developer could save time by using another developer's reusable component, rather than coding their own. The developer avoids both the work of writing the syntax and designing the logic of the component. The developer can design their own reusable components, keeping all the logic in one place, which can then be tested thoroughly. However, despite reuse being an important practice in software engineering, there is still a limited focus on this practice when it comes to low-code development platforms (LCDP).

%There are many different ways to do reuse in software engineering. Software libraries and frameworks are good examples of software that are intended to be reused. Developers may also scour the internet for things such as open-source software, or code snippets from websites like StackOverFlow, which can be reused. 

%%--from here: what reuse is there for low-code? What research? IN GENERAL
%%-- Then, what tools? examples of how tools tackle reuse
%%-- Then, what tools for blocks and robots? How do they tackle reuse? What are their strengths/weaknesses/limitations? MAKE SURE TO MENTION BY NAME!!
%%-- Show the tool we'll be working on (what does it do for reuse, and weaknesses).
%%-- Based on all that, what is our idea? What gap is it, we're addressing and how?
%% ---------------------
A study from 2021 studied several low-code platforms (LCPs), in order to identify characteristic features of LCPs. The identified features were presented according to how frequent they occured, with domain-specific reference artifacts being categorized as 'rare'. Most studied systems offered catalogs of "reusable functions or examples of predefined processes", but they were found to be generic, or have a limited scope\cite{LowCodePlatform}. This lack of focus on promoting reuse may impact the so-called 'Citizen Developers', who have little or no coding knowledge, and whom may then miss out on the benefits of reuse.

%%Maybe move SevenDevPlatforms up here?
There have been proposed some ideas on how to promote reuse for LCPs, such as the strongly-typed rich templating language OSTRICH, developed for the model-driven low-code platform OutSystems\cite{OSTRICH}. OutSystems provides scaffolding mechanisms for common development patterns and sample screen templates, both designed by experts on domain-specific languages (DSL).
The practice of using templates in the OutSystems platform involves cloning and modifying samples, which may require more knowledge than the end-user posseses.
The goal of OSTRICH is to remove this need for adapation when using templates, to remove the knowledge-barrier when making use of the available templates. This is done by abstracting and parameterizing the templates. 
A limitation of OSTRICH, is that it currently only supports the top nine most used production-ready screen templates from OutSystems. The end-user may not create and save their own templates, nor can they re-apply a template which they have customized. 
% type of experts NOT defined! page 1

Another approach focused on enabling reuse of models, by converting and merging models into a single graph (the Knowledge Graph), which acts as a repository of models\cite{ReuseKnowledgeGraphs}. This graph is used to provide recommendations to the end-user, based on the model they're currently building. While this feature of recommending models (either constructed by domain experts and then developed by model experts, or made by the end-user themselves) could prove very useful, the study is clearly not focused on guiding the user towards reusing their own models.

%Another approach focused on enabling model reuse by converting and merging heterogeneous models together into several graphs, which are then merged into one single graph (The Knowledge Graph), which acts as the repository of models. The Knowledge Graph can be queried to predict the next modeling step, based on the model being constructed by the user. This approach focuses on how to store, query, recommend and integrate the pre-defined models effeciently. End-Users can also persist their own models to the repository for later reuse.\\
%For citizen developers, this feature of recommending models which have been constructed by domain experts and then developed by model experts could prove very useful. However, while the user may persist their own models, the study is clearly not focused on guiding the user towards reusing their own models.\\ 
%On the other hand, some existing LCDPs offer the user the ability to create their own models - for example by defining a new block in a block-based tool\cite{ScratchExample}.

%having a graph serve as the model repository, created by converting and merging heterogeneous models together\cite{ReuseKnowledgeGraphs}. This approach aims to enable model reuse by merging the graphs into a single graph (the Knowledge graph), and then querying the Knowledge Graph to predict the next modeling step, based on the model being constructed by the user. This approach focuses on how to store, query, recommend and integrate the pre-defined models effeciently. End-Users can also persist their own models to the repository for later reuse. 
%Low code platform tools that focus on reuse
Building on the ideas discussed for improving reuse in low-code development platforms (LCDPs), 
several popular tools show these concepts in action. 
For instance, Webflow\cite{webflow} is a leading low-code platform that offers a wealth of features for building responsive websites. 
One of its standout features is the ability to create reusable components and UI kits, which can significantly 
speed up the development process. With Webflow’s intuitive interface, 
developers can quickly design and prototype components, and then reuse them across multiple pages and projects. 
Despite all of the useful features that this tools has, it does not 
provide guidance to the end-users to create custom reusable components. % which is the key feature of our project.

In a similar way, Mendix\cite{mendix} takes this further for full enterprise apps by offering 
shareable building blocks like simple actions (microflows) and UI parts that anyone 
on a team can grab and use again without recoding. Through its Marketplace, a free online hub, 
you can download ready templates, connectors for tools like Salesforce, and basic setups 
that fit right into new projects, making everything faster and more uniform. 
This approach builds on the flexibility seen in platforms like Webflow, but adds 
strong team tools and AI suggestions to spot and create reusable pieces, empowering 
even beginners to build complex apps while keeping reuse simple and widespread. This tool does offer guidance for the end-users to create custom reusable components through its AI suggestions,
a lot of times these suggestions are not accurate enough (how do we know this??**).

OutSystems\cite{outsystems} further enhances the concept of reuse in low-code development platforms by emphasizing rapid application delivery through its robust set of features.
Like Webflow and Mendix, OutSystems also provides a library of reusable components and templates that help developers complete projects faster. 
Its user-friendly visual development environment allows users to easily drag and drop elements while connecting with existing systems. 
OutSystems also supports teamwork with built-in version control and feedback features, making it easy for teams to share and improve 
reusable components. Additionally, the platform uses AI to suggest the best solutions and components for specific tasks. By encouraging reuse at both individual and team levels, OutSystems enables organizations to create scalable applications quickly while ensuring quality and consistency. Similarly to the previous tool explained, the AI suggestions that this tool provides are not always accurate to successfully guide the end-user to create custom reusable components (again, how do we know this?**).



In order to analyze how block-based robotics environments address reuse area, 4 representative platforms were compared: mBlock, MakeCode, SPIKE LEGO, VEXcode GO and Open Roberta. The comparison focused on three main dimensions of reuse: structural reuse (through user-defined blocks or functions), social reuse (through sharing or remixing existing projects), and interoperable reuse (through import/export capabilities).

\begin{table}[H]
  \small   
  \caption{Block Based Robotics Environments Reuse Support}
  \label{tab:reuse_support}
  \begin{tabular}{lcccc}
    \toprule
    Platform & Structural Reuse & Social Reuse & Interoperable Reuse & Reuse Support \\
    \midrule
    VEXcode GO    & X & X &  & Medium \\
    mBlock        & X & X & X & Medium \\
    MakeCode      & X & X & X & Medium \\
    Spike Lego    & X &  & X & Low \\
    Open Roberta  &  & X &  & Low \\
    \bottomrule
  \end{tabular}
\end{table}


In this context, “reuse support” represents a scale that measures how effectively each platform facilitates reuse-related features. High reuse support indicates that users can easily create, share, and adapt existing components or projects. Medium reuse support suggests that some reuse mechanisms are available but limited in scope or flexibility. Low reuse support implies that the platform provides only minimal or restricted features to promote reuse and improve user productivity.

As shown in Table 1, although these platforms include reusability features, they are quite limited, as none of them provide users with clear guidance on how to use these tools effectively, which restricts their ability to fully leverage them.


%Research also indicates that block based programming environments should guide the end users towards good code organization as many may lack the necessary knowledge or may become stuck due to errors.\cite{ChallengesInBlockBasedEnvironments} Although block based programming tools like Blockly were invented to teach programming to beginners by simple examples, Mayr-Dorn et al. mention that it is possible to express even large and highly complex real-world robot programs with the language concepts offered by these kind of block-based tools. \cite{blockBasedLanguagesForRobotProgramming} 

Lin and Weintrop (2021) noted that most existing research on block-based programming focuses on supporting the transition to text-based languages rather than exploring how features within BBP environments \cite{Lin2021Landscape}—such as abstraction or reuse—can enhance learning outcomes. In contrast, our work emphasizes guided abstraction, helping users understand and practice modular design directly within block-based environments.

Techapalokul and Tilevich (2019) proposed extending the Scratch programming environment with 
facilities for reusing individual custom blocks to promote procedural abstraction and improve 
code quality. They observed that while Scratch enables remixing of entire projects, it lacks 
mechanisms for reusing smaller, modular pieces of code. Their work suggests that supporting such 
fine-grained code reuse could enhance programmer productivity, creativity, and learning outcomes. 
Building on this idea, our project applies similar principles within the OpenRoberta environment 
by automating the detection of duplicate code segments and guiding users toward creating reusable 
custom blocks.
Adler et al. (2021) introduced a search-based refactoring approach to improve the readability of Scratch programs by automatically applying small code transformations, such as simplifying control structures and splitting long scripts. Their findings demonstrated that automated refactoring can significantly enhance code quality and readability for novice programmers. Building upon this concept, our project applies similar principles in the OpenRoberta environment, focusing on detecting duplicate code segments and guiding users toward creating reusable custom blocks to promote modularity and abstraction.\cite{Adler2021Improving}.

Existing block-based environments provide mechanisms for reuse, but lack intelligent support to help users recognize and apply reuse in practice. To address this gap, our project introduces a guided reuse assistant within the Open Roberta Lab environment.
The tool is designed to help users identify and apply reuse more easily while creating their robot programs. It works by automatically scanning a user’s block-based program to detect repeated code segments in the workspace. The system visually highlights the found duplicates, drawing the user’s attention to patterns that could be simplified.

The tool also offers the functionality to create the custom block for the end-user, by identifying the small differences between the repeated parts—such as numbers, variables, or parameters—and turning these differences into inputs for the new block. The tool automatically replaces all relevant duplicate sequences with the new custom block.

%When repeated blocks are detected, the assistant suggests creating a reusable custom block (function). It then helps the user generate this new block by identifying the small differences between the repeated parts—such as numbers, variables, or parameters—and turning these differences into inputs for the new block. After the user confirms, the system automatically replaces all the repeated sequences with calls to the newly created reusable block.

By combining ideas from procedural abstraction (organizing code into meaningful, reusable parts) and automated refactoring (improving code through intelligent transformations), our tool aims to make block-based programming more structured and efficient.
It encourages users to build programs that are modular and easier to maintain, helps reduce unnecessary repetition, and supports learning by making the concept of reuse clear and hands-on.
%In summary, our work bridges the gap between existing theoretical approaches to software reuse and their real-world application in block-based programming environments. Through this guided and semi-automated approach, we aim to make reuse visible, understandable, and practical for end-users working in Open Roberta.
\section{Study Design}




\subsection{Problem Investigation}
\subsubsection{Problem Context and Motivation}
%Problem investigation should be more specific to the chemistry end users. So we need problems that could happen in the chemistry lab and not generic problems of reuse.
End-user development (EUD) for collaborative robots (cobots) presents unique challenges, 
particularly for users without formal programming training. In domains such as chemistry 
laboratories, educational robotics, and industrial settings, end-users need to program 
robots to perform specific tasks but often lack the software engineering knowledge to 
write maintainable, well-structured code.
In the domain of Chemistry, one of the most prevelant and important tasks is performing experiments in labs in order to test a hypothesis, or to aid in the understanding of how chemicals react. 
Robots can be used in chemistry labs to automate experiments with great effect, as many experiments involve steps that are repetitive, and suspectible to human error - such as a step being overlooked, instructions being misread, etc. Automation of menial tasks will leave the chemists with more time for other work, and also comes with the added bonus of chemists not having to handle dangerous chemicals. 

One critical challenge in EUD is code reuse. Users frequently create repetitive code because they struggle to recognize duplicate patterns, lack knowledge about abstraction mechanisms, or find existing tools too complex to use effectively. This problem manifests in several ways: programs become unnecessarily long and difficult to maintain and small changes require modifications in multiple locations, increasing the risk of errors. Several  visual programming environments, like OpenRoberta Lab, don't provide assistance in identifying when code should be reused or how to extract repeated sequences into reusable components. As lab work in chemistry involves many repetitive tasks, these challenges can easily become an obstacle for the chemists, which may turn them away from using cobots, as the inconvenience outweighs the benefits.
%and users miss opportunities to learn fundamental programming concepts such as modularity and abstraction. - not relevant for chemists?


\subsubsection{Stakeholder Analysis}

\begin{itemize}
    \item \textbf{Chemistry Laboratory Personnel:} Chemists and lab technicians who use 
    cobots for repetitive tasks such as sample preparation, dispensing, mixing, and quality 
    control procedures. They possess deep domain expertise in chemistry but limited 
    programming knowledge, often creating long, repetitive programs that become difficult 
    to maintain when adapting experimental protocols. Their primary need is to quickly 
    create and modify robot programs without becoming programming experts.
\end{itemize}

\subsubsection{Artifact Requirements}
% make paragrapg of table 2 the need behaind each ine of this requiremnts. whta the solutioon? we should talk about what must be done and should.
\begin{table}[H]
  \small   
  \caption{Functional and Non-Functional Requirements}
  \label{tab:requirements}
  \begin{tabular}{p{1.5cm}p{0.8cm}p{8cm}p{1.3cm}}
    \toprule
    \textbf{Type} & \textbf{ID} & \textbf{Description} & \textbf{Priority} \\
    \midrule
  Functional 
    & FR1 & Detect duplicate/similar block sequences & High \\
    & FR2 & Visually highlight detected duplications & High \\
    & FR3 & Suggest creation of reusable custom blocks & High \\
    & FR4 & Allow users to accept/reject suggestions & High \\
    \midrule
  Non-Functional
    & NFR1 & Seamless Open Roberta Lab integration & High \\
    & NFR2 & Intuitive interface for end users & High \\
    & NFR3 & No interference with existing workflow & High \\
    & NFR4 & Clear visual feedback during detection & High \\
    \bottomrule
  \end{tabular}
\end{table}
The artifact requirements can be seen in table \ref{tab:requirements}. 
% FR1 the core of our tool. We do the work for the user, who doesn't have knowledge about practices of reuse.
% FR2 the more visible functions are, the more likely it is that users will be able to know what to do next. - lecture 1 review of HCI. 
% FR3 same as FR1. We make the option of a custom block more visible to the end-user. Also check lecture 2 about Reuse. 
% FR4 We want to give users the freedom to choose. The tool should support the end-user, it should not become an obstacle that they cannot avoid. Allows for customization (lecture 1 and 2).

% NFR1 an objective of EUD. We don't want to disrupt existing workflow. See 'objectives of EUD' lecture 2. 
% NFR2 Also objectives of EUD. 
% NFR3 Basically the exact same as NFR1 (or is at least also 'covered' by NFR1). Should we perhaps delete this requirement?
% NFR4 ...

\subsection{Treatment Design}

Our treatment focuses on developing a guided reuse assistant for the OpenRoberta Lab environment.
The purpose of this tool is to help users recognize which parts of their robot programs can be reused, 
and to make it easier for them to create reusable custom blocks. By doing this, we aim to reduce 
repetitive code and help users learn important programming concepts such as modularity and abstraction.

\subsubsection{Overview of the Tool}

The guided reuse assistant is built as an extension inside Open Roberta Lab, which uses the Blockly 
framework. The assistant runs directly in the web browser and interacts with the user’s block workspace. 
Its main job is to look through the user’s program, find repeated sequences of blocks, and guide 
the user in turning them into reusable blocks.

The tool works in three main steps:
\begin{enumerate}
    \item \textbf{Detecting Repeated Code:} 
    The assistant automatically scans the user’s program and searches for parts that look the same
     or very similar. These are marked as potential duplicates.

    \item \textbf{Highlighting and Suggesting Reuse:} 
    Once duplicates are found, the system highlights them in the workspace and shows a message 
    suggesting that these sections could be made into a reusable block (function). 
    This helps users see repetition they might not have noticed before.

    \item \textbf{Helping the User Create a New Block:} 
    If the user agrees to the suggestion, the assistant opens a small guide to help them create 
    the new block. It automatically detects any small differences between the repeated parts, 
    such as numbers or variable names, and turns them into inputs (parameters) for the new block. 
    When the block is created, repeated code is replaced by the new reusable block.
\end{enumerate}
% -------------------------------------------------------------------------------------------------------------------------------------------------------------------
% validity threat here or in discussion?
% Treatment validation needs to be split into subsections "data gathering and analysis" How we gather the data and how we analyze them
%"Task execution" We need to explain what exactly will be the task
\subsection{Treatment Validation}
The treatment validation for this study adopts a mixed-methods evaluation approach to assess 
the effectiveness of the proposed features for guiding users in creating custom reusable components (blocks) within 
the OpenRoberta environment. 

\subsubsection{Participant Recruitment}
A total of 10 participants will be selected to ensure a diverse range of experience levels with block-based programming. 
Time constraints and resource availability have influenced the decision to limit the number of participants.
Participants will be recruited from a diverse pool of individuals affiliated with the University of Southern Denmark and the broader chemistry community. 
This group of participants includes chemistry teachers, professional chemical engineers, and students currently enrolled in chemistry-intensive curricula. 
To ensure relevant practical expertise, the selection specifically targets those who frequently engage in laboratory environments. 
The experimental sessions will be conducted across a range of environments to accommodate participant availability. 
Physical sessions will take place within the chemistry laboratories at the University of Southern Denmark (SDU) as well as a private residential setting. 
For remote participants, sessions will be administered virtually using Discord for communication and AnyDesk for remote desktop control.

\paragraph{Ethical Considerations and Sampling}
Prior to the commencement of the study, all participants are required to sign a consent form acknowledging their voluntary participation and granting permission for screen recording and data usage. 
It should be noted that this recruitment strategy constitutes \textit{convenience sampling}. As such, they may not represent the general population.
% -------------------------------------------------------------------------------------------------------------------------------------------------------------------
\subsubsection{Task Execution}
The participants will initially be given a short introduction to the OpenRoberta UI, as well as the mujoco robot simulator. 
They will then perform one task which is described by a set of pre-defined steps to perform. This task has been specifically designed to promote the reusability aspect.
The task is focused on the domain of chemistry, as it is modelled after a real lab experiment perfomed by chemistry students at SDU. 

The participants will be instructed to program the robot to execute the following sequence of operations:

\begin{enumerate}
    \item Move the robot arm above mix cylinder
    \item Mix the chemistry ingredients
    \item Move the robot arm above the analysis pad
    \item Analyze the sample
    \item If the solution is analyzed (use if statement) then show a response message in the laptop’s screen 
    \item Place the following three objects into their corresponding slots in the chemistry equipment toolbox:
    \begin{itemize}
        \item Methanol cylinder
        \item Chloroform syringe
        \item Toluene syringe
    \end{itemize}
    \item Important notes for the participants:
    \begin{itemize}
        \item \textit{After placing an object to its slot in the toolbox \textbf{wait 2 seconds} before you move to pick a new one.}
        \item \textit{After placing the \textbf{chloroform syringe} to its slot, \textbf{move the robot arm up by 10 cm} before you move to pick the next chemistry object }
        \item \textit{Click the \textbf{play} button on the bottom right corner to start the simulation}
        \item \textit{Click the \textbf{reset} button on the bottom right corner to reset the scene of the robot simulator}
    \end{itemize}
\end{enumerate}

Most optimal solution pre-defined by the researchers:
 \begin{figure}[h]
     \centering
     \includegraphics[width=0.8\textwidth]{most optimal solution.png}
     \caption{The optimal solution implemented in OpenRoberta, utilizing a custom block for the object placement sequence.}
     \Description{Block-based program showing the optimal solution with a custom reusable block for placing chemistry objects into toolbox slots.}
     \label{fig:optimal_solution}
 \end{figure}

Instead of creating a long linear sequence of blocks (hard-coding the movement for all three objects), the most optimal solution utilizes a **Custom Reusable Component** to handle the repetitive action of placing an object to its corresponding slot inside the equipment toolbox. 
This approach not only reduces redundancy but also enhances code maintainability and readability, aligning with best practices in software development.

All the participants will try to complete the task using both the standard and the enhanced version of OpenRoberta. 
Half of the participants will begin using the enhanced version of OpenRoberta, while the other half will start with the standard version.
Participants’ interactions with the platform will be observed throughout the task.
Guidance will be provided from the researchers to the participants throughout the task.

% -------------------------------------------------------------------------------------------------------------------------------------------------------------------
\subsubsection{Data Gathering and Analysis}
Data collection focuses on both quantitative performance and qualitative feedback from participants:
\begin{enumerate}
    \item \textbf{Task Completion Time:} Comparing the participants who will first use the enhanced version of OpenRoberta against those who will first use the standard version.
    \item \textbf{Solution Accuracy:} Evaluated by comparing the participant's block configuration against the pre-defined optimal solution.
    \item \textbf{Qualitative Feedback:} Collected via a post-experiment survey designed to capture demographic data and subjective perceptions of the utility of the block creation guidance features.
\end{enumerate}

This comprehensive evaluation will provide a detailed understanding of how useful and effective is the block creation guidance feature to the end-users.
% -------------------------------------------------------------------------------------------------------------------------------------------------------------------
% In this section, you should present the results obtained during the treatment validation. The section should be organized around the research questions, presenting detailed results and key findings for each one of the RQs
The treatment validation concluded with 10 participants in total. The results show that all participants 
preferred the enhanced version of OpenRoberta Lab compared to the standard version, with 25\% of participants 
finding the enhanced version to be 'better' than the original, and 75\% found it to be 'much better'. 
[insert piechart]. Results also showed that 75\% of participants found the enhanced version 'easy' to use 
and 25\% finding it 'very easy'.\\
%we need the time the users took to finish their tasks, and 'accuracy' in creating blocks. (did all users decide to use the tool to create blocks when presented with the choice? Did Dimitris instruct them to do it, or did they perform the task without any interference from observers?)
% analyse the qualitative results
% results about the highlights
Results showed a high level of satisfaction with the highlights, with 87,5\% of users being 'satifisfied' or 
'very satisfied', while 12,5\% felt neutral about the highlights. 
% suggestions for changes
\section{Results}
% -----------------------------------------------------------------------------------------
% OVERVIEW
% -----------------------------------------------------------------------------------------
The treatment validation was concluded with a total of 10 participants. 
The analysis of the collected data combines quantitative metrics regarding user preference and satisfaction 
with qualitative feedback derived from survey responses.


\subsection{Performance Evaluation}
To evaluate the efficiency and effectiveness of the proposed reusable component features, 
we analyzed two primary metrics: Task Completion Time and Solution Accuracy.

% -----------------------------------------------------------------------------------------
% TASK COMPLETION TIME
% -----------------------------------------------------------------------------------------
\subsubsection{Task Completion Time}
The total time required to complete the experimental task was recorded for both the \textit{Standard} and \textit{Enhanced} conditions. 

We compared the performance of participants based on the order of conditions (see Table \ref{tab:time_comparison_vertical}).
The analysis reveals a significant reduction in task duration when using the Enhanced 
version. The average completion time for the participants that used the Enhanced version first was $8.5$ 
minutes, compared to $10$ minutes for the Standard version. 
\begin{equation}
    \text{Efficiency Improvement} = \frac{10.0 - 8.5}{10.0} \times 100\% = 15\%
\end{equation} 



%\begin{itemize}
 %   \item \textbf{Enhanced OpenRoberta version First:} Participants who started with the Enhanced version completed the task in an average of 8 and a half minutes.
 %   \item \textbf{Standard OpenRoberta version First:} Participants who started with the Standard version completed the task in an average of 10 minutes.
%\end{itemize}
\begin{table}[h]
    \centering
    \caption{Breakdown of Mean Task Completion Times}
    \label{tab:time_comparison_vertical}
    % 'm{10cm}' allows the text to wrap and centers it vertically
    % 'c' centers the number vertically
    \begin{tabular}{m{10cm} | c} 
        \hline
        \textbf{Experimental Condition} & \textbf{Mean Time (min)} \\
        \hline
        % No '\\' here. Text and number are on the same row.
        \textit{Group of Participants that used the Enhanced OpenRoberta Version First} & 8.5 \\
        \hline
        \textit{Group of Participants that used the Standard OpenRoberta Version First} & 10.0 \\
        \hline
    \end{tabular}
\end{table}

% -----------------------------------------------------------------------------------------
% SOLUTION ACCURACY 
% -----------------------------------------------------------------------------------------
\subsubsection{Solution Accuracy}
Solution accuracy was evaluated by comparing participant solutions against the optimal reference 
solution defined in the treatment evaluation.

\paragraph{Adoption of Reusable Blocks}
A key metric was the voluntary adoption of the custom reusable component. 
In the \textit{Enhanced} version, $10/10$ participants successfully implemented a custom reusable block to handle the repetitive object placement steps. 
In contrast, in the \textit{Standard} condition, participants predominantly relied on linear, repetitive code structures.
Without the guidance features, none of them recognized the opportunity to create a reusable block.

% -----------------------------------------------------------------------------------------
% SURVEY QUANTITATIVE RESULTS
% -----------------------------------------------------------------------------------------
\subsection{Survey Quantitative Results}

\subsubsection{User preference between Standard and Enhanced Versions of OpenRoberta}
The survey results indicate a unanimous preference for the enhanced version of the OpenRoberta Lab. 
As illustrated in Figure \ref{fig:compare_versions}, 70\% of participants rated the enhanced version 
as ``much better'' than the standard version, while the remaining 30\% rated it as ``better.'' 
No participants preferred the standard version or rated the two versions as equivalent.

\begin{figure}[h]
    \centering
    \includegraphics[width=\textwidth]{survey_comp_enh_sta.png}
    \caption{Summary of participant responses regarding overall preference between the standard and enhanced versions of OpenRoberta}
    \label{fig:compare_versions}
\end{figure}
\begin{figure}[h]
    \centering
    \includegraphics[width=\textwidth]{usability_of_feature.png}
    \caption{Summary of participant responses regarding overall preference between the standard and enhanced versions of OpenRoberta}
    \label{fig:usability_of_feature}
\end{figure}
\begin{figure}[h]
    \centering
    \includegraphics[width=\textwidth]{hl_eval.png}
    \caption{Summary of participant responses regarding overall preference between the standard and enhanced versions of OpenRoberta}
    \label{fig:hl_eval}
\end{figure}
\begin{figure}[h]
    \centering
    \includegraphics[width=\textwidth]{hl_pref.png}
    \caption{Summary of participant responses regarding overall preference between the standard and enhanced versions of OpenRoberta}
    \label{fig:hl_pref}
\end{figure}

\subsubsection{Usability of the Guidance Feature}
Regarding usability of the enhanced OpenRoberta version, we received high acceptance scores. As illustrated in Figure \ref{fig:usability_of_feature}, 40\% of participants found the 
enhanced version ``very easy'' to use, and 60\% rated it as ``easy.''
No participants rated the enhanced version as ``Neither easy nor difficult,'' ``Difficult,'' or ``Very difficult'' to use.
\subsubsection{Evaluation of the Visual Highlighting}
A key component of the enhanced version was the visual highlighting designed 
to guide the user into an automatic custom reusable block creation. 
As shown in Figure \ref{fig:hl_eval}, results showed a high level of user satisfaction, with 90\% of participants reporting they were either ``satisfied'' 
(20\%) or ``very satisfied'' (70\%) with the features. Only one participant (10\%) expressed a neutral stance.

\subsubsection{Visual Highlighting Style Preference}
When asked about specific highlighting preferences, as depicted in Figure \ref{fig:hl_pref} the \textit{Animated Color Highlight} was the most popular choice, 
preferred by 50\% of the users. A significant portion of participants (30\%) expressed no strong preference 
between the styles, suggesting that the presence of guidance was more important than the specific animation style 
used.

% -----------------------------------------------------------------------------------------
% SURVEY QUALITATIVE RESULTS
% -----------------------------------------------------------------------------------------
\subsection{Qualitative Feedback}
The post-experiment survey included open-ended questions to gather detailed feedback. The thematic analysis of 
these responses revealed two primary findings:

\paragraph{Efficiency and Speed}
When asked to identify the biggest difference between the two versions, the majority of participants cited 
\textit{efficiency}. Responses frequently described the enhanced version as ``faster'' and noted that it 
``saved a lot of time.'' This aligns with the quantitative preference data, suggesting that the reusability 
features successfully reduced the perceived workload.

\paragraph{Suggestions for Improvement}
Participants also provided constructive feedback regarding the function blocks. Two participants specifically 
suggested that the system should more clearly \textit{``specify parameter names''} within the function blocks to 
improve clarity. Another participant noted that the function call block should be pre-configured for immediate 
use in the blockchain. These suggestions highlight a need for clearer labeling in future iterations of the 
interface.

%----------------------------------------------------------------------------------------------------------------------
\section{Discussion}
%In this section, you should discuss the results presented in the previous section. The discussion should be organized around lessons learned (5.1), implications for practice (5.2), and threats to validity (5.3). The whole discussion must be grounded on the existing literature and argue how your contributions advance the state-of-the-art in the chosen EUSE area
\subsection{Lessons Learned}
Based on the feedback from the participants, as well as observations of how they solved the task, the participants found the enhanced version of OpenRoberta Lab to be better than the standard version. Noteably, 7 out of 8 participants commented on how the enhanced version let them perform their task faster. As described in section 2, this is also one of the main benefits of reuse in the field of software engineering. 
%need to review the screen recordings as well!! How was the users' behavior when using the enhanced tool? How long does it take them to insert a custom block into the workflow, after they've clicked 'yes' to letting our tool create one for them? Did they seem confused or hestitant when they used the tool for the first time?
While a somewhat large(?) percentage of the participants had no preference in regards to the visual look of the highlight, half of the users picked the 'Animated Color Highlight'. This suggests that dynamic visuals - in this case: the blocks changing color repeatedly - are well-suited for catching the user's attention. 
% Another thing we should refine in section 3 - we should prolly explain more about why we designed the tool the way we did. Need to back it up with research. 


Changes suggested by the participants mainly focus on smaller customizations of the tool and the OpenRoberta Lab UI. It would be amiss to claim that the lack of suggested changes, focused on the tool overall, indicate that there is no need for improvement of the tool. As many of the participants consider themselves 'beginners' in regards to Computer Programming, it's likely that they lack ideas about other ways the tool could have been designed. Instead, these answers can be interpreted as the participants having little to no issue with the current design. 

\subsection{Implications for Practice}
% From what I could find, this section is basically about the potential effects/significance of our study's findings. How can the results we found influence real-life practices and research? Basically smth like; participants showed positive reactions to our tool, finding it both easier to use, but also faster. Our solution/idea is worth doing further research about, as the results suggest that our idea can successfully help end-users perform reuse. 

\subsection{Threats to Validity}
\subsubsection{Convenience Sampling} The participants to the study were either aquantiances of one of the authors of the study, or were recruited through these aquantiances. As such, the results of this study do not represent the general population within the domain of chemistry. 
\subsubsection{Limitations to observation} Due to constraints with time and flexibility, only on the the authors was present to observe the participants. To ensure that data from the observation was not affected by this, a screen recording of each participant performing the task was saved. Several of the authors reviewed and discussed these recordings together to extract data.
%...other stuff?



\section{Citations and Bibliographies}

The use of \BibTeX\ for the preparation and formatting of one's
references is strongly recommended. Authors' names should be complete
--- use full first names (``Donald E. Knuth'') not initials
(``D. E. Knuth'') --- and the salient identifying features of a
reference should be included: title, year, volume, number, pages,
article DOI, etc.

The bibliography is included in your source document with these two
commands, placed just before the \verb|\end{document}| command:
\begin{verbatim}
  \bibliographystyle{ACM-Reference-Format}
  \bibliography{bibfile}
\end{verbatim}
where ``\verb|bibfile|'' is the name, without the ``\verb|.bib|''
suffix, of the \BibTeX\ file.

Citations and references are numbered by default. A small number of
ACM publications have citations and references formatted in the
``author year'' style; for these exceptions, please include this
command in the {\bfseries preamble} (before the command
``\verb|\begin{document}|'') of your \LaTeX\ source:
\begin{verbatim}
  \citestyle{acmauthoryear}
\end{verbatim}


  Some examples.  A paginated journal article \cite{Abril07}, an
  enumerated journal article \cite{Cohen07}, a reference to an entire
  issue \cite{JCohen96}, a monograph (whole book) \cite{Kosiur01}, a
  monograph/whole book in a series (see 2a in spec. document)
  \cite{Harel79}, a divisible-book such as an anthology or compilation
  \cite{Editor00} followed by the same example, however we only output
  the series if the volume number is given \cite{Editor00a} (so
  Editor00a's series should NOT be present since it has no vol. no.),
  a chapter in a divisible book \cite{Spector90}, a chapter in a
  divisible book in a series \cite{Douglass98}, a multi-volume work as
  book \cite{Knuth97}, a couple of articles in a proceedings (of a
  conference, symposium, workshop for example) (paginated proceedings
  article) \cite{Andler79, Hagerup1993}, a proceedings article with
  all possible elements \cite{Smith10}, an example of an enumerated
  proceedings article \cite{VanGundy07}, an informally published work
  \cite{Harel78}, a couple of preprints \cite{Bornmann2019,
    AnzarootPBM14}, a doctoral dissertation \cite{Clarkson85}, a
  master's thesis: \cite{anisi03}, an online document / world wide web
  resource \cite{Thornburg01, Ablamowicz07, Poker06}, a video game
  (Case 1) \cite{Obama08} and (Case 2) \cite{Novak03} and \cite{Lee05}
  and (Case 3) a patent \cite{JoeScientist001}, work accepted for
  publication \cite{rous08}, 'YYYYb'-test for prolific author
  \cite{SaeediMEJ10} and \cite{SaeediJETC10}. Other cites might
  contain 'duplicate' DOI and URLs (some SIAM articles)
  \cite{Kirschmer:2010:AEI:1958016.1958018}. Boris / Barbara Beeton:
  multi-volume works as books \cite{MR781536} and \cite{MR781537}. A
  presentation~\cite{Reiser2014}. An article under
  review~\cite{Baggett2025}. A
  couple of citations with DOIs:
  \cite{2004:ITE:1009386.1010128,Kirschmer:2010:AEI:1958016.1958018}. Online
  citations: \cite{TUGInstmem, Thornburg01, CTANacmart}.
  Artifacts: \cite{R} and \cite{UMassCitations}.

\section{Acknowledgments}

Identification of funding sources and other support, and thanks to
individuals and groups that assisted in the research and the
preparation of the work should be included in an acknowledgment
section, which is placed just before the reference section in your
document.

This section has a special environment:
\begin{verbatim}
  \begin{acks}
  ...
  \end{acks}
\end{verbatim}
so that the information contained therein can be more easily collected
during the article metadata extraction phase, and to ensure
consistency in the spelling of the section heading.

Authors should not prepare this section as a numbered or unnumbered {\verb|\section|}; please use the ``{\verb|acks|}'' environment.

\section{Appendices}

If your work needs an appendix, add it before the
``\verb|\end{document}|'' command at the conclusion of your source
document.

Start the appendix with the ``\verb|appendix|'' command:
\begin{verbatim}
  \appendix
\end{verbatim}
and note that in the appendix, sections are lettered, not
numbered. This document has two appendices, demonstrating the section
and subsection identification method.

\section{Multi-language papers}

Papers may be written in languages other than English or include
titles, subtitles, keywords and abstracts in different languages (as a
rule, a paper in a language other than English should include an
English title and an English abstract).  Use \verb|language=...| for
every language used in the paper.  The last language indicated is the
main language of the paper.  For example, a French paper with
additional titles and abstracts in English and German may start with
the following command
\begin{verbatim}
\documentclass[sigconf, language=english, language=german,
               language=french]{acmart}
\end{verbatim}

The title, subtitle, keywords and abstract will be typeset in the main
language of the paper.  The commands \verb|\translatedXXX|, \verb|XXX|
begin title, subtitle and keywords, can be used to set these elements
in the other languages.  The environment \verb|translatedabstract| is
used to set the translation of the abstract.  These commands and
environment have a mandatory first argument: the language of the
second argument.  See \verb|sample-sigconf-i13n.tex| file for examples
of their usage.

\section{SIGCHI Extended Abstracts}

The ``\verb|sigchi-a|'' template style (available only in \LaTeX\ and
not in Word) produces a landscape-orientation formatted article, with
a wide left margin. Three environments are available for use with the
``\verb|sigchi-a|'' template style, and produce formatted output in
the margin:
\begin{description}
\item[\texttt{sidebar}:]  Place formatted text in the margin.
\item[\texttt{marginfigure}:] Place a figure in the margin.
\item[\texttt{margintable}:] Place a table in the margin.
\end{description}

%%
%% The acknowledgments section is defined using the "acks" environment
%% (and NOT an unnumbered section). This ensures the proper
%% identification of the section in the article metadata, and the
%% consistent spelling of the heading.
\begin{acks}
To Robert, for the bagels and explaining CMYK and color spaces.
\end{acks}

%%
%% The next two lines define the bibliography style to be used, and
%% the bibliography file.
\bibliographystyle{ACM-Reference-Format}
\bibliography{sample-base}


%%
%% If your work has an appendix, this is the place to put it.
\appendix

\section{Research Methods}

\subsection{Part One}

Lorem ipsum dolor sit amet, consectetur adipiscing elit. Morbi
malesuada, quam in pulvinar varius, metus nunc fermentum urna, id
sollicitudin purus odio sit amet enim. Aliquam ullamcorper eu ipsum
vel mollis. Curabitur quis dictum nisl. Phasellus vel semper risus, et
lacinia dolor. Integer ultricies commodo sem nec semper.

\subsection{Part Two}

Etiam commodo feugiat nisl pulvinar pellentesque. Etiam auctor sodales
ligula, non varius nibh pulvinar semper. Suspendisse nec lectus non
ipsum convallis congue hendrerit vitae sapien. Donec at laoreet
eros. Vivamus non purus placerat, scelerisque diam eu, cursus
ante. Etiam aliquam tortor auctor efficitur mattis.

\section{Online Resources}

Nam id fermentum dui. Suspendisse sagittis tortor a nulla mollis, in
pulvinar ex pretium. Sed interdum orci quis metus euismod, et sagittis
enim maximus. Vestibulum gravida massa ut felis suscipit
congue. Quisque mattis elit a risus ultrices commodo venenatis eget
dui. Etiam sagittis eleifend elementum.

Nam interdum magna at lectus dignissim, ac dignissim lorem
rhoncus. Maecenas eu arcu ac neque placerat aliquam. Nunc pulvinar
massa et mattis lacinia.

\end{document}
\endinput
%%
%% End of file `sample-manuscript.tex'.
